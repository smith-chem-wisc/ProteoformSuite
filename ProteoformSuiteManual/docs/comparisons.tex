%!TEX root = ../proteoform_suite_manual.tex
%---------------------------------------------------------------------
%	COMPARISONS
%---------------------------------------------------------------------

\section{Proteoform Comparisons}
Proteoforms observations are identified either by comparing with theoretical proteoforms or with other experimental observations that have already been identified. First, ET mass differences, and then EE mass differences.
\\

Forming peaks

\subsection{Experimental-Theoretical Comparisons}
\subsection{Experimental-Experimental Comparisons}

\subsection{Histogram}
\begin{itemize}
	\item What to look for
	\item How to extend mass range of considered mass differences
	\item How to view raw ET histogram, what that means
	\item How to view decoy histograms and what that means
\end{itemize}

\subsection{Peak Selection}
\begin{itemize}
	\item Peak width option. Separate for ET and EE.
	\item Peak Threshold
	\item Peak FDR, how it's calculated, how important it is relative (max FDR possible for proteoforms connected by that mass difference, actual FDR is lower)
	\item Automatic peak selection (upcoming)
\end{itemize}

\subsection{Curating Relations}
\begin{itemize}
	\item Can accept or reject individual relations to curate families
	\item Table filters
\end{itemize}


