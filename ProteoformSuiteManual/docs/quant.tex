%!TEX root = ../proteoform_suite_manual.tex
%---------------------------------------------------------------------
%	QUANTIFICATION
%---------------------------------------------------------------------

\section{Quantification}
Using 1:1 NeuCode ratio for quant, and having bundled those observations into experimental proteoforms using mass and retention time, we can now do relative quantification of the 

\subsection{Requirements for Quantification}
\begin{itemize}
	\item Minimum bioreps from any single condition, explain
	\item Minimum bioreps from any condition, explain
	\item Minimum bioreps from each condition, explain
	\item Only these are considered for imputation, quantification
\end{itemize}

\subsection{Adjusting Background Imputation}
\begin{itemize}
	\item Example plot, what you're looking for, namely that the ``background projected'' is neatly in the shoulder of the observed gaussian
	\item Background shift, explained (stdevs from observed mean)
	\item Background Width, explained (stdevs)
\end{itemize}

\subsection{Volcano Plot}
\begin{itemize}
	\item Example plot, what you're looking for
	\item What to change if it looks awry
\end{itemize}

\subsection{FDR Determination by Permutation Analysis}
using Tusher et al. methodology (cite)
\begin{itemize}
	\item Example plot, what you're looking for (steep, levels off, steep -- significant, insigificant, significant)
	\item How to adjust the offset to change which values are accepted. How to do that, namely, just isolating that middle portion of the line.
\end{itemize}

\subsection{GO Term Analysis}
What the three categories of GO terms are. GO terms are interdependent because of heirarchy. We use a dependent p-value analysis (benjimini-yekeulti, cite). We are required to use protein-level GO terms because there are no proteoform-level ones, although PTMs and the like surely affect the function of the molecules.
\begin{itemize}
	\item How to choose a background GO term set.
	\item Quantified set is the most conservative, considers only those meeting quant requirements. Often gives no results.
	\item Detected set is all identified proteoforms
	\item Can specify a flat file with UniProt accession numbers on each line for a custom background list. These must be included in the theoretical database.
	\item Theoretical set. All theoretical proteoforms. This is very liberal and gives many significant results.
\end{itemize}

\subsection{Cytoscape Displays}
\begin{itemize}
	\item Example visualization (quant, with pie charts)
	\item How the annuls and bolded text is used to highlight significant differences. Checkboxes available if that's not desired.
	\item Build all families, explain
	\item Build all families w/ significant change, explain
	\item Build selected families, explain
	\item Build all families with significant go terms, explain
	\item Build all families with Selected go terms, explain
\end{itemize}
